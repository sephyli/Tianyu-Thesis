% !Mode:: "TeX:UTF-8"
\chapter{绪论}

\section{课题背景及目的}

计算机视觉是计算机通过分析图像来获取图像中信息的能力,具有极其广泛的应用,如人脸识别、自动驾驶、视觉导航、医学影像分析等。计算机视觉领域的发展与人类生活的各个领域息息相关,而现如今,通过多样的机器学习技术提升计算机视觉系统的分析能力,成为当今计算机视觉研究领域亟待解决的问题。

对许多计算机视觉领域的经典任务而言,卷积神经网络(Convolutional Neural Networks, CNN)在多年的发展中取得了良好的成果。CNN针对网格式数据设计,在图片这样的矩阵型数据上具有很强的拟合能力,从而可以在很大程度上提取图像中像素级的信息和特征。因此,构建不同网络结构的CNN,并通过执行图像分类\upcite{He2016DeepRL},目标检测\cite{Ren2015FasterRT}等视觉任务,验证CNN提取特征的能力,一直以来都是本领域研究的热点。

\begin{figure}
    \centering
    \includegraphics[width=.8\textwidth]{figure/visual.png}
    \caption{例子}
    \label{fig:relation_sample}
\end{figure}

但是在目标检测、语义分割、区域分类等任务中,图像中的物体在种类和数目上都较为复杂,物体间也具有多样化的关系。多样的物体种类和有限的数据使很多不常见的分类仅仅有很少的训练样例,而传统的视觉模型往往只对物体所在的局部区域进行分析,那么模型在训练样例较少的分类上的表现就十分糟糕。如果模型能学习到图像的全局上下文信息,就可以借助其他常见分类与稀有分类间的关系,来提升模型对稀有分类的表现。

在识别场景中的物体时,人类往往能通过多种关系来做出推理判断。如图\ref{fig:relation_sample},这个办公桌的场景在生活中十分常见。当人一看到办公桌上的显示器,就会自动从记忆中调取信息,进而得知显示器附近会存在键盘、鼠标等物体。与人类很自然地通过常识性语义关系或空间关系辅助认知不同,传统的视觉模型缺少这样视觉推理的能力,也就是通过利用图像中的额外信息进行推理,增强视觉模型识别物体的能力。

本研究基于图像数据,构建图结构的知识图谱(Knowledge Graph, KG)来保存物体之间的关系。近年提出的图卷积网络(Graph Convolution Networks, GCN)可以有效提取图结构上的特征,并完成网络节点分类等任务。因此,本研究将通过GCN和知识图谱分析图像物体间的关系信息,结合传统CNN对图像像素信息的提取能力,设计一种效率更高的新型视觉推理模型,以获得更好的区域分类效果,从而满足计算机视觉研究在自动驾驶等领域应用上的需求。

\section{国内外研究现状}

\subsection{图卷积网络}

一些传统的图神经网络对图的邻接矩阵进行分析,使用邻接矩阵、特征(Feature)和权重矩阵做矩阵乘法运算。而卷积的引入是对图的拉普拉斯(Laplacian)矩阵进行卷积运算,使该运算具有信息传播能力,更切合信息流动的本质。图上的卷积运算往往分为频域(Spectral Domain)和空域(Spatial Domain)两种。基于频域运算的图卷积网络[4-5]使用特殊构造的卷积核,对图进行傅里叶变换,从而在频域进行运算。基于空域的图卷积网络[6-7]直接将卷积操作定义在图上,直接对某个节点的近邻节点进行操作。

一种先进的图注意力网络(Graph Attention Networks, GAT)在空域图卷积网络的基础上,受机器翻译中的联合学习[17]启发,引入了注意力机制,通过自注意力策略计算每个节点和其近邻节点的一种隐式表达,可以应用在具有任意度的图节点上,且在Cora,Citeseer等图神经网络的测试集上取得了近年来最好的效果。

\subsection{利用知识图谱和图神经网络的计算机视觉研究}

在图像分类,视觉推理,语义分割等任务中,均有许多研究利用了图结构来提升视觉任务的表现。在目标检测(Object Detection)任务中,图神经网络用作感兴趣区域(Region of Interest, RoI)的特征提取[10,11]。在视觉推理的区域分类任务中,X.Chen等的超越卷积的迭代视觉推理系统[9](Iterative Visual Reasoning Beyond Convolutions)将区域间的空间关系和语义关系构建成图,利用图神经网络进行分析。上述研究均取得了超过普通卷积网络的成果。

该系统由两个模块组成,局部模块(Local Module)和全局模块(Global Module)。局部模块主要使用空间内存(Spacial Memory)实现,可以提取图像中的像素级信息,全局模块则是基于图结构的关系推理。该模型在全局模块中通过有效利用人类常识性知识图谱来直接为语义关系提供参考,且显式地标识出区域与区域间的空间关系,更好地利用了这些关系信息。

全局模块的处理方法利用了图神经网络,基于图结构且具有对关系的分析能力,但该模块并非图卷积网络。模块内的关系图主要由三个部分组成,区域图(Region Graph)、知识图(Knowledge Graph)和分配图(Assignment Graph)。其节点分为区域节点(Region Nodes)和类节点(Class Nodes),区域节点表示一图像中的某一区域,类节点表示在数据集中出现的每一种类别。

但该系统并没有使用图卷积网络,对邻接矩阵和特征矩阵的密集连接使其有较高的计算成本,也难以控制参数数量。因此,要结合图卷积网络对知识图谱进行分析,尝试取得更有效的成果。


\section{课题研究方法}

\section{论文构成及研究内容}
