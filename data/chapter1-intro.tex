% !Mode:: "TeX:UTF-8"
\chapter{绪论}

\section{课题背景及目的}

计算机视觉是计算机通过分析图像来获取图像中信息的能力,具有极其广泛的应用,如人脸识别、自动驾驶、视觉导航、医学影像分析等。计算机视觉领域的发展与人类生活的各个领域息息相关,而现如今,通过多样的机器学习技术提升计算机视觉系统的分析能力,成为当今计算机视觉研究领域亟待解决的问题。

对许多计算机视觉领域的经典任务而言,卷积神经网络(Convolutional Neural Networks, CNN)在多年的发展中取得了良好的成果。CNN针对网格式数据设计,在图片这样的矩阵型数据上具有很强的拟合能力,从而可以在很大程度上提取图像中像素级的信息和特征。因此,构建不同网络结构的CNN,并通过执行图像分类\upcite{He2016DeepRL}等视觉任务,验证CNN提取特征的能力,一直以来都是本领域研究的热点。

\begin{figure}
    \centering
    \includegraphics[width=.8\textwidth]{figure/visual.png}
    \caption{本图中的办公桌上存在着多种物体间的语义关系}
    \label{fig:relation_sample}
\end{figure}

但是在目标检测\upcite{Ren2015FasterRT}、区域分类\upcite{Chen2018IterativeVR}、语义分割等任务中,图像中的物体在种类和数目上都较为复杂,物体间也具有多样化的关系。多样的物体种类和有限的数据使很多不常见的分类仅仅有很少的训练样例,还有严重遮挡、模糊等现象。而传统的视觉模型往往只对物体所在的局部区域进行分析,那么模型在这种较稀有分类上的表现就十分糟糕。如果模型能学习到图像的全局上下文信息,就可以借助其他常见分类与稀有分类间的关系,来提升模型在稀有分类上的表现。

在识别场景中的物体时,人类往往能通过多种关系来做出推理判断。如图\ref{fig:relation_sample},这个办公桌的场景在生活中十分常见。当人一看到办公桌上的显示器,就会自动从记忆中调取信息,进而得知显示器附近会存在键盘、鼠标等物体。与人类很自然地通过常识性语义关系或空间关系辅助认知不同,传统的视觉模型缺少这样视觉推理的能力,也就是通过利用图像中的额外信息进行推理,增强视觉模型识别物体的能力。

本研究基于图像数据,构建图结构的知识图谱(Knowledge Graph, KG)来保存物体之间的关系。近年提出的图卷积网络(Graph Convolution Networks, GCN)可以有效提取图结构上的特征,并完成网络节点分类等任务\upcite{Kipf2017SemiSupervisedCW}。因此,本研究将通过GCN和知识图谱分析图像中物体间的关系信息,结合传统CNN对图像像素信息的提取能力,设计一种效率更高的新型视觉推理模型,以获得更好的区域分类效果,从而满足计算机视觉研究在自动驾驶等领域应用上的需求。

\section{国内外研究现状}

\subsection{视觉知识库}

计算机视觉领域的快速发展离不开公开大规模数据集的完善,从最简单的对物体对象的标注\upcite{Lin2014MicrosoftCC, Russakovsky2015ImageNetLS},到对场景信息的标注\upcite{Zhou2018SemanticUO},到对关系信息的标注\upcite{Krishna2016VisualGC}等等,上述数据集的建立使依赖大量数据的卷积神经网络更易于训练,也间接定义了对卷积神经网络的评判标准。

\subsection{利用关系信息的计算机视觉研究}

在深度学习成为计算机视觉的主流方法之前,已经有很多传统的视觉算法利用了对象间关系来提升算法表现。Divvala等的研究\cite{Divvala2009AnES}使用对象间的共存关系来对检测到的对象进行重新评分,从而提升目标检测的成绩,也有一些研究指出对象间的空间关系可以改善图像分割的效果\cite{Galleguillos2008ObjectCU, Gould2008MultiClassSW}。

随着深度学习在视觉领域的广泛应用,在目标检测,区域分类,语义分割等任务中,均有许多研究利用了关系信息来提升视觉任务的表现\cite{Hu2017RelationNF, Jiang2018HybridKR, Chen2018IterativeVR, Xu2019ReasoningRCNNUA}。

\subsection{利用知识图谱的计算机视觉研究}

最近有一些研究将知识图谱和图神经网络结合用于计算机视觉问题。Li等的研究\cite{Li2017SituationRW}提出了一种基于图神经网络的情景识别方法,可以有效地分析角色间的联合依赖关系。一些研究\cite{Wang2018ZeroShotRV, Kampffmeyer2019RethinkingKG}利用知识图谱来解决难度较高的零样本识别任务。Yang等的研究\cite{Yang2018VisualSN}提出了一个视觉导航框架,该框架可以动态地对知识图谱进行更新,结合语义关系使视觉导航可以更好地泛化到未见场景。

\subsection{图卷积网络}

图神经网络\cite{Scarselli2009TheGN}是对结构化数据进行特征提取的有效网络结构,根据传播方式的不同,可细化为图卷积网络等不同形态。图卷积网络中卷积的引入使运算更切合信息在拓扑结构上传播的本质,这类卷积运算分为频域(Spectral Domain)和空域(Spatial Domain)两种。基于频域运算的图卷积网络\cite{Bruna2014SpectralNA, Kipf2017SemiSupervisedCW}使用特殊构造的卷积核,对图进行傅里叶变换,从而在频域进行运算。基于空域的图卷积网络\cite{Duvenaud2015ConvolutionalNO,Atwood2016DiffusionConvolutionalNN}将卷积操作定义在节点上,直接对其近邻节点进行操作。

一种先进的图注意力网络\cite{Velickovic2018GraphAN}(Graph Attention Networks, GAT)在空域图卷积网络的基础上,受机器翻译中的联合学习\cite{Bahdanau2015NeuralMT}启发,引入了注意力机制,通过自注意力策略计算每个节点和其近邻节点的一种隐式表达,可以应用在具有任意度的图节点上。


\section{课题研究方法}

\section{论文构成及研究内容}
