% !Mode:: "TeX:UTF-8"
\chapter*{总结与展望}
\addcontentsline{toc}{chapter}{总结与展望}

本视觉推理模型在VG-1000上的区域分类任务上具备着良好的提升效果,但模型对关系的利用建立在初始分类较为准确的情况下,在更难的任务下,如VG数据集中前三千类构成的VG-3000数据集上,基线每分类准确率可能仅在$12\%$甚至更低,这意味着对于稀有分类,模型的软分类策略的有效性会相对下降,导致模型直接应用在较难任务上的提升效果可能会有所衰减。

未来,本模型可以通过同义词集,直接拓展到COCO、ADE等数据集上,对于这样相对更简单的数据集,模型理应具备良好的效果。另外,模型目前针对区域分类任务,对更为通用的目标检测任务的支持还有待检验。模型可以针对目标检测任务进行改进,割离GCN模块对目标检测中区域框位置回归预测的影响,使目标检测模型的区域位置回归与区域分类两个任务分为两个分支,从而将视觉推理能力专注在区域分类部分中,以期提升传统目标检测模型的能力。

另一方面,本视觉推理模型对知识图谱的有效性十分依赖,说明知识图谱很大程度上影响了区域特征传播时的方向与权重分配。对此,尝试通过更有效的方式构建知识图谱或许可以进一步加强模型的视觉推理能力。如从现实世界中真实存在的人类常识性关系构建知识图谱,而非VG数据集中标注的关系来构建。现实世界中的语义关系多种多样,分布更加均匀而广泛,只要构建合适的同义词集将其关系与数据集中的类别进行对应,知识图谱的效果可能会更好。另外,本视觉推理模型在利用关系知识图谱时,使用其关系出现的频率以及对象间的空间距离作为关系强弱程度的判断标准,与区域特征通过关系传播时的影响因素。但关系出现的频率受限于数据集标注质量与统一性,在一定程度上没有真实反映物体间的关系强度。而基于带距离权重的物体共现频率,即同时在一张样例上出现且空间距离较近的一对物体出现的频率,可能是另一种构建知识图谱的思路。

以上基于统计的知识图谱构建方法都仰仗于数据集的丰富程度,尤其是对较少样本的丰富程度。或许,我们可以使用深度学习方法来构建知识图谱。给定物体的区域特征,预测出物体间是否具备关系。同样,使用VG数据集的标注信息作标签来训练这一网络,期望这一网络可以学习到区域特征间语义上的相关性,使用这一网络输出的结果作为区域特征传递的权重。本研究所使用的注意力机制意图在一定程度上隐式地学习到区域间的关系的强弱,而这一网络或许可以显式地学习到这一性质,从而提升模型对关系的利用程度。

\cleardoublepage
