% !Mode:: "TeX:UTF-8"
\chapter{网络结构与推理框架}

本章节介绍组成视觉推理模型的整体框架。该框架由CNN模块和GCN模块组成,首先介绍CNN模块的主干设计,随后介绍知识图谱的构建和生成,最后介绍GCN模块的设计以及GCN更新区域特征的方法。

\section{CNN模块的构建}

CNN模块的输入是图像及对象的无标签区域坐标。首先,经由CNN模块的卷积网络主干,提取图像的全局像素特征,然后使用感兴趣区域层,将全局特征依不同对象的区域坐标转化为区域特征,并将区域特征输出到下一模块。除提取特征之外,CNN模块也要做出对各区域分类的初步预测,以将区域特征与不同分类的节点相对应。

本模块使用了多种技术提升特征提取的能力,意图对大部分常见分类,生成较为准确的初始预测。使用这样的预测将区域特征输入GCN模块,GCN模块才可以更好地将区域特征在相应类别的节点间传播。

\subsection{卷积网络主干}

卷积网络主干的设计至关重要,这一部分主要负责提取图像的全局像素特征。在本研究所针对的区域分类任务中,包含所有对象的高分辨率图像直接输入到这一卷积网络进行降维,输出一个通道数较多,而特征图大小较小的高维特征。

受Mask R-CNN启发,本模块使用ResNet作为卷积网络主干,这一模型具备极强的特征提取能力,在以往的计算机视觉研究中得到了充分验证\upcite{He2016DeepRL, He2017MaskR}。ResNet使用残差连接,解决了卷积神经网络在层数较深时的梯度丢失问题,避免了网络的加深反而导致模型拟合能力变差的情况,这一设计使更深的、拟合能力更强的网络成为可能。本研究通过实验对比,选择了101层的ResNet以获得最好的特征提取能力,为模型学习到较难分类的特征提供基础。

本模块所使用的ResNet经过多层降维后,输出特征图到卷积网络主干尾部的特征金字塔网络(Feature Pyramid Networks, FPN)\upcite{Lin2017FeaturePN}。这一部分从ResNet的不同阶段接受不同尺寸的输入,从而生成结合不同维度特征的高质量特征。下面介绍ResNet的技术实现细节。

\subsubsection{ResNet}

ResNet提出了残差连接和瓶颈结构,两者皆可以用以构建更深更高效的卷积网络。前者可以解决在网络较深时出现的梯度丢失问题,后者则利用瓶颈结构使模型更小更高效。

\paragraph{残差连接}

在以往的卷积网络中,采用直接堆叠卷积核的方式加深网络,而在深层网络中,往往会出现梯度丢失问题,即在过多层的梯度传导中,梯度降低至0。这导致加深网络的深度反而导致网路的性能下降。对于本研究来说,只有够深的网络才能有较强的特征提取能力,使模型具备视觉推理的能力。而ResNet解决了这一问题,其在每个卷积的构建单元中,增加一个残差分支来传导梯度,防止梯度在过多卷积层的传导中丢失。这使得加深卷积网络,如使用100层以上的网络时,网络的性能依然可以继续提升。

\begin{figure}[h!]
    \centering
    \includegraphics{figure/ResNet_building_block.eps}
    \caption{一个残差连接中的卷积构建单元}
    \label{fig:resnet_building_block}
\end{figure}

残差连接的捷径分支有多种连接方法,如投影、零填充和恒等连接。

\subparagraph{恒等连接} 

等式(\ref{res:iden})代表一个卷积构建单元,其中$\mathbf{x}$代表输入的特征,,$\mathcal{F}$代表一个由卷积核构成的残差函数,往往由两层$3 \times 3$卷积层构建而成。而$W_{i}$为该构建单元中卷积核的权重。如图\ref{fig:resnet_building_block}.

\begin{equation} \label{res:iden}
    y = \mathcal{F} (\mathbf{x}, \{W_{i}\}) + \mathbf{x}.
\end{equation}

\subparagraph{投影连接} 

等式(\ref{res:proj})中的$W_{s}$为一个线性的投影函数,一般使用$1 \times 1$卷积核来实现。线性体现在该卷积核后不会增加激活函数,而非线性激活操作会在完成加法运算后统一进行。

\begin{equation} \label{res:proj}
    y = \mathcal{F} (\mathbf{x}, \{W_{i}\}) + W_{s} \mathbf{x}.
\end{equation}

在实际应用中ResNet在对特征图进行降维时,每个卷积构建单元的输入和输出维度不同,此时使用恒等连接会使两个维度不一样的特征向量相加,造成错误。则应使用投影连接,将单个卷积构建单元中捷径分支的维度与残差分支,即卷积核主体的维度相统一。但此时,统一特征维度所使用的卷积操作为密集连接,一定程度上会增加模型的大小。相对来说,恒等连接不会增加模型的参数和计算的复杂度。因此在卷积构建单元无降维操作时,直接使用恒等连接是更优的选择。

本模型在使用ResNet时,101层的网络中存在5次降维操作,即模型的大部分残差连接都由恒等连接构建而成。

\subparagraph{零填充} 
零填充也是在残差分支与捷径分支维度不同时的一种统一维度的方法。它直接在捷径分支的输出向量要增加的维度上填充0来匹配输出,但它的效果并不如投影连接好,仅作为在运算性能受限的情况下提供的一种构建选项。

\paragraph{瓶颈结构}
我们需要更深的网络来获得更好的特征提取能力的同时,模型的参数数量也需要控制,否则训练成本就会无限制地上升。

\begin{figure}[h!]
    \centering
    \includegraphics[width=.8\textwidth]{figure/ResNet_bottleneck.eps}
    \caption{A deeper residual function $\mathcal{F}$}
    \label{fig:resnet_bottleneck}
\end{figure}

因此,ResNet提出了瓶颈结构(Bottleneck)作为网络的卷积构建单元。如图~\ref{fig:resnet_bottleneck},对于每个残差函数$\mathcal{F}$,其使用了三个卷积层来代替原有的两层,分别是$1 \times 1$,$3 \times 3$和$1 \times 1$卷积核,其中$1 \times 1$卷积核负责降低然后恢复维度,使$3 \times 3$卷积核作为瓶颈,可以具有较小的输入和输出,大大降低了模型的参数数量和运算成本。另外,经过实验证明,应用了瓶颈结构后,模型的准确率和运算速度双双提高。

无参数的恒等连接对于瓶颈结构尤为重要。如果在瓶颈结构中全使用投影连接,会导致模型的参数数量和运算成本翻倍。这是因为捷径连接到两个$1 \times 1$卷积核的高维端,较高的特征维度若使用$1 \times 1$卷积核这样较为密集的运算,就会带来巨大的运算成本。因此恒等连接和瓶颈结构是相辅相成的设计。

\subsection{感兴趣区域}

输入图像经由卷积网络主干提取全局像素特征后,输出的特征图最小为原分辨率的$1/32$。由于输入图像的分辨率不定,且图像中对象区域的大小也互不相等,则输出的特征图本身的大小就参差不齐,但GCN模块接受的输入应当是固定大小的特征向量。而传统的全连接层针对固定大小的输入设计,无法应用于本研究所处理的区域分类任务。因此,本模块加入了RoI层对各个大小不一的区域的特征进行提取。

\section{知识图谱构建}

在使用GCN传播区域特征之前,还需要构建出用于视觉推理的知识图谱$G$,该知识图谱包含不同对象类别之间的人类常识性语义关系与空间关系。知识图谱在模型中起到关键作用,因为它关系着不同类别间的特征沿怎样的路径传播。

\subsection{使用语义关系构建知识图谱}

构建基于语义关系的知识图谱有多种途径。首先,根据VG数据集中提供的标注中出现的对象间关系,如位置关系,主-谓-宾关系等等,结合该关系在对象类别之间共同出现的频率,可以构建出依据关系的知识图谱$G_{R}$。另外,根据VG数据集提供的标注中出现的对象属性,如颜色、材质、形状等等,可以构建出对象类别与属性间的概率分布表\upcite{Xu2019ReasoningRCNNUA}。则对每一对对象类别$c_i$和$c_j$的概率分布$P_{c_i}$与$P_{c_j}$计算Jensen–Shannon散度,就可以得到属性知识图谱$G_{A}$的边$e_{c_{i}, c_{j}}^{A}=J S\left(P_{c_{i}} \| P_{c_{j}}\right)$。

$G_{R}$与$G_{A}$中部分分类间的子图如图\ref{fig:subgraph}所示。当特征在关系知识图谱上$G_{R}$传播时,高置信度的常见分类的节点,就可以通过知识图谱的边对与其相关联的稀少分类节点传播特征,从而丰富稀少分类节点中的特征信息,从而强化其置信率。而与其无关联的节点不会接收到特征的传递,防止特征在分类间泛化,降低模型表现。另外,通过常识得知,属性相近的物体在外观上近似。因为颜色和材质决定了物体表面对光线的反射水平,颜色和材质相近的物体在图像中的像素颜色近似,而外形则影响图像中的形状。如图\ref{fig:subgraph:a}中,摩托车和自行车具有极其相近的属性,其外观也极其相近。而外观相近的物体间存在误认情况,通过属性知识图谱$G_{A}$,使属性相近的物体的分类节点间有特征的传递,可以使预测的置信度趋向由误认分类向正确分类移动。

\begin{figure}[h!]
    \centering
    \subfigure[基于关系的知识图谱]{
        \includegraphics[width=0.45\textwidth]{figure/subgraph_r.png}
        \label{fig:subgraph:r}
    }
    \subfigure[基于属性的知识图谱]{
        \includegraphics[width=0.45\textwidth]{figure/subgraph_a.png}
        \label{fig:subgraph:a}
    }
    \caption{$G_{R}$与$G_{A}$中部分分类间的子图图示}
    \label{fig:subgraph}
\end{figure}

\subsection{使用空间关系构建知识图谱}

根据图像中对象的空间位置关系,可以构建出基于空间的知识图谱$G_{Spatial}$。对象的位置关系分为上下、左右,以及对象区域的包含、重叠等关系。对于对象的上下左右关系,知识图谱边的权重由对象在图像中的像素距离决定,像素距离越远,对象之间的关系越稀疏,因此边的权重应该越低。因此,本研究在节点上应用了核函数$\kappa(x)=\exp (-x / \Delta)$(其中$\Delta = 50$),如图\ref{fig:kernel_func}。对于对象区域的包含或重叠关系,使用图像的交并比(Intersection over Union, IoU)来定义空间关系的边,即对象区域重叠的面积与区域并集面积的比值,IoU越大,代表对象关系越密切。

\begin{figure}
    \centering
    \includegraphics[width=.5\textwidth]{figure/kernel.png}
    \caption{核函数$\kappa(x)=\exp (-x / \Delta)$}
    \label{fig:kernel_func}
\end{figure}



\section{GCN模块}