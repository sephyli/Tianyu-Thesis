% !Mode:: "TeX:UTF-8"

% 中英文摘要
\begin{cabstract}
本研究提出了一个基于GCN知识图谱分析的视觉推理模型,该模型弥补了传统的使用卷积网络的目标检测模型仅利用图片的像素信息对图像进行分析的不足,使用图卷积网络对人类常识性知识图谱中的物体关系进行分析,将图片以外的多种语义信息融入模型,使模型具有视觉推理的能力。本模型由两个主要模块组成,分别是CNN模块和GCN模块:CNN模块由普通卷积网络结合感兴趣区域层构建而成,具备着极强的提取图像像素特征的能力;GCN模块由图注意力层构建而成,可以使区域特征在图结构的知识图谱上进行传播。最终模型可以结合CNN模块与GCN模块的预测结果,对物体进行重新分类以提升准确率。其中知识图谱的构建使用到了多种关系信息,包括物体的常识性语义关系、物体属性的相关度和物体的空间关系三种。相比于由最新的使用卷积网络的目标检测模型构建而成的基线模型,本视觉推理模型表现出了远超于这一基线模型的物体分类能力。在VG测试集上的区域分类任务中,样例平均精度(AP)比基线模型取得了$4.25\%$的绝对提升。
\end{cabstract}

\begin{eabstract}

\end{eabstract}